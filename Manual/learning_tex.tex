\documentclass[11pt,letter,oneside]{report}
\begin{document}

\title{Space Invaders}
\author{Peter Hamilton and Jared Pilcher}
\date{Fall 2011}
\maketitle

\tableofcontents

\chapter{Space Invaders Overview}
\section{History of Space Invaders}

Space invaders was a really popular game in the early 16th century.  Children played it out in the fields with the trebuchet when their parents were out of town.

Designed by Tomohiro Nishikado, Space Invaders is one of the greatest arcade games of all time.  It hold the Guinness World Record for greatest arcade game.
 

\section{Game Play}

\subsection{Objective}

The world is being invaded by marching aliens.  All of humanity has gathered behind 4 bunkers for protection.  You control the only weapons available, a supply of 3 tanks.  The fate of the world lies in your hands.

\subsection{The Tank}

You are the tank.

As the tank you must dodge bullets.

\subsection{Enemies}

There are 55 enemies on the screen.

\subsection{Points}

\section{Game Details and Specifications}

\subsection{Aliens}

\subsubsection{Alien Placement and Appearance}
There are 55 enemies that start on the screen.  They are on a 11 by 5 grid.  Each grid is 15 pixels by 15 pixels.  There are three types of aliens.  The first two rows are the widest aliens, the next two rows are a little narrower and the top row has the narrowest alien.

\subsubsection{Alien Movement}
The aliens move laterally in increments of 2 pixels.  When the rightmost aliens reach the right edge of the screen, the grid of aliens will move down 7 pixels and start moving left.  When the leftmost aliens reach the left edge of the screen, they will move down 7 pixels and start moving right.  The rightmost alien and leftmost alien is the closest alien to the respective sides that has not been killed.

\subsection{The Tank}
The tank moves laterally across the bottom of the screen.  The user should be able to hold down the left or right buttons to 

\subsection{Bullets}
\subsubsection{Tank Bullets}
Tank bullets, unlike Alien Bullets, only have one appearance.  When a bullet is fired, it should appear on the screen just above the center of the tank.  When a bullet is on the screen, the tank cannot fire again until the bullet has left the screen.
\subsubsection{Alien Bullets}
There are two types of alien bullets:
\begin{description}
\item{Zig Zag Bullets}  Zig zag bullets rotate cyclicly through the following following appearances:

\begin{enumerate}
\item <
\item S
\item >
\item 2
\end{enumerate}

\item{Cross Bullets}  Cross bullets oscillate through the following appearance:

\begin{enumerate}
\item T
\item +
\item L
\end{enumerate}

\end{description}

There can only be 4 bullets on the screen at any given point in time.  Once a bullet leaves the screen, a new bullet can be fired.

\end{document}
